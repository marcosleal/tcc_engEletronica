\begin{resumo}
	
	No resumo deve-se ressaltar de forma clara e sintética a natureza e o objetivo do trabalho, o método que foi empregado, os resultados e as conclusões mais importantes, seu valor e originalidade. O resumo é a “apresentação concisa dos pontos relevantes de um texto. Constitui elemento essencial em textos de natureza técnico-científica” (ASSOCIAÇÃO BRASILEIRA DE NORMAS TÉCNICAS, 2003, p.3). O resumo não pode	ultrapassar 250 palavras. Abaixo do resumo devem aparecer as palavras-chave (mínimo três, máximo cinco, separadas por ponto final e iniciadas com letra maiúscula).
	
	\textbf{Palavras-chave}: mínimo três. máximo cinco. separadas por ponto final e iniciadas com letra maiúscula.
\end{resumo}