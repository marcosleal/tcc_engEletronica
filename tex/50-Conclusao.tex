\chapter{Considerações Finais}
    
    Nessa parte apresenta-se a síntese interpretativa dos principais argumentos usados, mostrando se os objetivos foram atingidos e se a(s) 31 hipótese(s) foi(foram) confirmada(s) ou rejeitada(s). Também se podem incluir recomendações e/ou sugestões para trabalhos futuros. Deve-se fazer uma rápida retomada dos capítulos que compõem o trabalho e uma espécie de autocrítica, fazendo um balanço a respeito dos resultados pela pesquisa.
    
    Atenção! A conclusão não constitui uma ideia nova ou um simples anexo sem importância ao trabalho. Pelo contrário, é nesse momento em que todas as ações do estudo são expostas, analisadas e finalizadas.
    
    Para melhor orientar-se, responda às seguintes questões:
    
    \begin{alineas}
	    \item a pesquisa resolve o problema, amplia a compreensão, mostra novas relações ou mesmo descobre outros problemas em relação ao originalmente escolhido?
	    \item a hipótese, ao final, foi confirmada ou refutada pela pesquisa?
	    \item os objetivos geral e específicos previamente definidos foram alcançados?
	    \item a metodologia de trabalho escolhida foi suficiente para a consecução de seus propósitos? houve necessidade, ao longo da pesquisa, de adotar outras técnicas ou procedimentos para lidar com situações não previstas?
	    \item a bibliografia previamente selecionada correspondeu às suas expectativas?
	\end{alineas}
    