\chapter{Métodos aplicados}

	Mostra-se como serão executados a pesquisa e o desenho metodológico que se pretende adotar: será do tipo quantitativa, qualitativa, descritiva, explicativa ou exploratória. Será um levantamento, um estudo de caso, uma pesquisa experimental, por exemplo. Segundo Gil 2006), a organização da metodologia depende do tipo de pesquisa a ser realizada. No entanto, alguns elementos devem ser apresentados, como:
	
	\begin{alineas}
	\item tipo de pesquisa: se é de natureza exploratória, descritiva ou explicativa.
	Convém, ainda, esclarecer acerca do tipo de delineamento a ser adotado pesquisa experimental, levantamento, estudo de caso, pesquisa bibliográfica);
	
	\item  população e amostra: envolve informações acerca do universo a ser estudado, da extensão da amostra e da maneira como será selecionada;
	\item coleta de dados: envolve a descrição das técnicas a serem utilizadas para a coleta de dados. Modelos de questionários ou testes deverão ser incluídos nessa parte. Se a pesquisa envolver técnicas de entrevista ou de observação, é também o momento de expor o assunto;
	\item análise dos dados: descrevem-se os procedimentos a serem adotados tanto
	para a análise quantitativa quanto qualitativa.
	\end{alineas}